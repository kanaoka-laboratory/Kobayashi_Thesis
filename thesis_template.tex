\documentclass[a4j]{jarticle}
\date{}
\usepackage[dvipdfmx]{graphicx}
\usepackage{plistings}

\begin{document}
\lstset{
  basicstyle={\ttfamily},
  identifierstyle={\small},
  commentstyle={\smallitshape},
  keywordstyle={\small\bfseries},
  ndkeywordstyle={\small},
  stringstyle={\small\ttfamily},
  frame={tb},
  breaklines=true,
  columns=[l]{fullflexible},
  numbers=left,
  xrightmargin=0zw,
  xleftmargin=3zw,
  numberstyle={\scriptsize},
  stepnumber=1,
  numbersep=1zw,
  lineskip=-0.5ex
}

%表題
\makeatletter %ここから\makeatotherまで触らなくていい
	\def\@thesis{令和2年度 東邦大学理学部情報科学科 卒業研究}
	\def\id#1{\def\@id{#1}}
	\def\department#1{\def\@department{#1}}
	
	\def\@maketitle{
		\begin{center}
			\vspace{10mm}
			{\large \@thesis \par}	%修士論文と記載される部分
			\vspace{50mm}
			{\huge\bf \@title \par}	% 論文のタイトル部分
			\vspace{15mm}
			{\Large 学籍番号 \@id \par}	% 学籍番号部分
			\vspace{5mm}
			{\Large \@author \par}	% 氏名
			\vspace{50mm}
		\end{center}
		\begin{flushright}
			{\large 金岡研究室}
		\end{flushright}
	}
\makeatother

\title{AndroidアプリケーションにおけるIPv4アドレスのハードコーディングに関する調査と分析} %中括弧の中に卒論のタイトル
\id{5517044} %学籍番号
\author{小林 裕} %名前
\maketitle{\title} %表紙の出力
\thispagestyle{empty} %このページ(表紙)のページ番号を消去
\newpage %強制改ページ

\tableofcontents %目次の出力





\newpage
\section{はじめに} 
インターネット上に接続された機器同士がデータをやりとりする際、ネットワーク上で通信相手を間違わないよう、それぞれを唯一に特定するために割り当てられいてる識別子Internet Protocol(以後IP)アドレスが存在する。

現在IP Version 4 (以後IPv4) が広く利用されているが、インターネットの利用者が増え続けたことによりIPv4アドレスが枯渇していることが問題になっている。

IPv4アドレスは32ビットのビット長を持ち、約43億個のアドレスを表現することができる。この数はインターネット黎明期の頃は十分な大きさと考えられていたが、インターネットに接続される機器が増え続けたことにより、割当の限界を迎えつつある。
そのためアドレス空間をIPv4から大幅に拡張するなど対策が取られたIP Version6 (以後IPv6) の導入がされ、
IPv4からIPv6への移行へと移行していくことになる。しかしながら、
IPv4アドレスとIPv6アドレスの間には互換性がなく、
相互通信を行うことが出来ないためIPv4アドレスとIPv6アドレスの混在環境が続き、
いずれ混在する環境からIPv6だけの環境 (IPv6 Single Stack環境) へと移行していく。

モバイル環境においては、AppleのiPhoneやiPadに提供されるアプリケーション (以後アプリ) ではIPv6に対応することが必須とされている[1]。一方、Android OS上ではそういった要件は示されていない。すでにIPv6 Single Stack環境からのアクセスにおいて、正しく動作しないAndroidアプリが複数発見されている。

過去2012年にはIETFにてIPv6 Only Networkについての情報共有がなされ[2]、2018年には北口らによってOS各種のIPv6対応状況調査が行われ[3]、その後、加茂によってAndroidに焦点を当てv6のみの環境でOSが稼働するか、マーケットが対応しているか、Androidアプリが動くかという複数の視点で行われた[4]。
その結果、古いOSでのIPv6未対応状況、マーケットのIPv6未対応状況、アプリの大部分がIPv6に対応していないことが分かった。しかし、Androidアプリが未対応である原因についてはまだ調査がされていなかった。

AndroidアプリがIPv6に対応していない原因を考えた場合、IPv4とIPv6の混在環境やIPv6 Single Stack環境で動くようにOSなどは対応されているため、
アプリ側で特定のIPバージョンが指定されていることからIPv6 Single Stack環境で動かないことが1つの原因として考えられる。
IPアドレスがIPv4であることを前提として指定されていた場合、IPv6 Single Stack環境ではそのアプリが動作しないだけではなく、
利用者が悪意のある攻撃者によってMan in the middle(中間者攻撃)を受けるリスクも存在する。
これらのことを考えると、IPv4のハードコーディングは避けられるべきであると考えられる。

そこで本研究では、Androidアプリを対象にアプリ内にIPv4アドレスの指定がされているのか調査と分析を行っていく。

中間発表の段階では、テスト環境として2520個のアプリを対象に、IPv4アドレスの指定が行われいるか調査したところ、2520個中241個のアプリにIPv4アドレスの記載があることが分かった。
更なる調査として記載があったIPv4アドレスの出現数を数えた結果、一番出現頻度が多かったIPv4アドレスは"127.0.0.1"のループバックアドレスであった。
本論文ではさらなる調査結果を元に、脅威や原因の考察をした。

本論文の構成は以下の通りである。第2章で本研究に関する技術についての解説を行い、第3章では関連研究について説明を行う。
第4章では本研究の調査内容と手法を述べ、第5章で調査結果を述べる。第6章で考察し、最後に第7章で本研究のまとめを述べる。


\newpage
\section{前提知識}
\subsection{Android}
AndroidとはGoogle社によって開発された携帯汎用オペレーティングシステムである。スマートフォンやタブレット、時計、テレビなどに搭載されている。

\subsubsection{アプリケーション配布マーケット}
アプリケーション配布マーケット(以後アプリマーケット)とはアプリを提供するマーケットのことで、
公式アプリマーケットはGoogleが提供するGoogle PlayやAppleが提供するApp Storeがある。
また、Androidアプリに関してはサードパーティ製のアプリマーケットも存在する。
マーケットからソフトウェアを入手しインストールするために、マーケットへのアクセス専用アプリが提供されている。Androidでは
Google Playアプリが提供されており、多くの端末においてあらかじめインストールされている。
\subsubsection{アプリ開発の言語と開発環境}
AndroidアプリはJava、Kotlin、C/C++といった言語で開発が可能である。開発環境としてAndroid Studioが公式に提供されている。
また、iOSアプリを開発する際は、Swift、Objective-Cといった言語で開発が可能である。開発環境としてXcodeが公式に提供されている。
\subsubsection{ライブラリ}
ライブラリとは、ある特定の機能をもったプログラムを他のプログラムから呼び出して利用できるように細分化し、それらをまとめて一つにしたものを呼ぶ。
また、それ単体ではプログラムとして動作させることはできないものを呼ぶx。
Android開発においては、Google社から提供されているライブラリやサードパーティライブラリが存在する。
\subsubsection{APK}
APKとは、Google社によって開発されたAndroid専用ソフトウェアパッケージのことで一種のファイルである。
APKの入手方法はAPKストアからダウンロードする方法や、単体で公開されているAPKファイルをダウンロードする方法などが存在する。
一般にAPKの拡張子は".apk"であり、内部の構造はzipファイルと同様である。

APKファイルに対してzipファイルと同様の解凍処理を行い、得られる内部フォルダは以下のような構成になっている。(●はフォルダ、○はファイルを表す。)
\begin{tabbing}
 \hspace{8mm} \= \hspace{10mm} \= \hspace{12mm} \= \hspace{15mm} \kill
\>┏●META-INF
\\\>┃\>┣〇MANIFEST.MF
\\\>┃\>┣〇*.SF
\\\>┃\>┗〇*.RSA or -*.DSA
\\\>┣●lib
\\\>┣●res
\\\>┣●assets
\\\>┣〇AndroidManifest.xml
\\\>┣〇classes.dex
\\\>┗〇resources.arsc
\end{tabbing}

\subsubsection{APKストア}
APKストアとは、開発者の作成したAndroidアプリの配信を代行するサービス、及びそれを行っているWebサイトのことである。
Androidの公式APKストアはGoogle Play1つのみであり、非公式のAPKストアは数多く存在する。
\subsubsection{Smaliファイル}
Smaliファイルとは、APKに含まれるソースコードをDalvikバイトコードとして表記したものであり、その情報はclasses.dexに含まれている。
Apktool等を用いてAPKより取得できるSmaliファイルは、Dalvikバイトコードで記述されている。
\subsubsection{APKからSmaliファイルの変換}
APKからSmaliファイルを変換する方法は、Apktoolを使う方法がある。ApktoolはAPKに含まれる全てのリソース・ソースコード・XMLを展開し
ソースコードのみをSmaliファイルとして変換が可能である。これにより、
ソースコード内にIPv4アドレスのハードコーディングが行われているのか調査することが可能である。
Apktoolによるリソース・ソースコード・XMLの展開で得られる内部フォルダは以下のような構成になっている。

\begin{tabbing}
	\hspace{8mm} \= \hspace{10mm} \= \hspace{12mm} \= hspace{15mm} \kill
 \\\>┏●assets
 \\\>┣●lib
 \\\>┣●original
 \\\>┣●res
 \\\>┣●smali
 \\\>┣●unknown
 \\\>┣〇AndroidManifest.xml
 \\\>┗〇apktool.yml
 \end{tabbing}



\subsection{Internet Protocol}
Internet Protocl(以後IPと呼ぶ)とは、通信ネットワークを相互に接続し、データ通信をするための通信規約である。
また、IPにおいてパケットを送受信する機器を判別するための番号をIPアドレスという。
現在IPはIPv4とIPv6が混在した状態にある。

IPv4アドレスは32ビットで構成される世界で重複のないアドレスで、8ビット毎4つにピリオドで区切った数値列を10進数で表され、
例えば"192.168.255.255"のように表記されます。
一方、IPv6アドレスは128ビットを16ビット毎8つにコロンで区切った数値列を16進数で表され、
例えば"2001:0DB8:0000:0000:0008:0800:200C:417A"のように表記されます。
IPv6アドレス表記では連続する0は省略可能であるが、各フィールドに少なくとも1つの数値を含ませなければならない。
また、16ビットの0が複数連続している場合は"::"と何処でも省略することができるが、省略できるのは一カ所のみである。
例えば"2001:0DB8:0000:0000:0008:0800:200C:417A"というIPv6アドレスは"2001:DB8::8:800:200C:417A"と省略する。

IPv4アドレスではネットワーク部とホスト部に別れて構成される。
一方、IPv6アドレスはサブネットプレフィックスとインターフェースIDに別れて構成される。
IPv6のサブネットプレフィックスは、IPv4のネットワーク部に該当し、IPv6のインターフェースIDは、IPv4のホスト部に該当する。
サブネットフィックスは64ビットが標準的な値であるため、インターフェースIDも64ビットであることが一般的である。

IPv4アドレスとIPv6アドレスの大きな違いとしては、アドレスビット長が32ビットから128ビットと大幅に拡張している事が挙げられる。
その他に、IPv4ではヘッダチェックサムがあるが、TCPやアプリケーション側でエラーチェックを行なっているためIPレベルのエラーチェックは不要だとして、IPv6ではヘッダチェックサムが削除された。
また、品質制御利用のためのフローラベルが追加されるなどの違いがある。

IPv4での通信の種類は、ユニキャスト、ブロードキャスト、マルチキャストの3種類でしたが、IPv6ではブロードキャストがマルチキャストへと統合された。
IPv4でブロードキャストで行われていたものは、IPv6ではマルチキャストを利用してい行われている。
一方、IPv6での通信の種類は、ユニキャスト、マルチキャスト、エニーキャストの3種類である。

\subsubsection{IPv4アドレス枯渇問題}
現在IPv4が広く利用されているが、インターネットの利用者が増えたことによりIPv4アドレスが枯渇していることが問題になっている。
IPv4アドレスは32ビットのビット長を持ち、約43億個のアドレスを表現することができる。この数はインターネット黎明期の頃は十分な大きさと考えられていたが、
インターネットに接続される機器が増え続けたことにより、割り当ての限界を迎えつつある。そのためアドレス空間をIPv4から大幅に拡張するなど対策のとられたIPv6が導入され、
IPv4からIPv6への移行が進んでいくと予想される。しかしながら、IPv4とIPv6の間には互換性がなく、相互通信を行うことができないため
IPv4とIPv6の混在環境が続き、いずれ混在する環境からIPv6だけの環境(IPv6 Single Stack)へと移行していく。

\subsubsection{IPv6 Single Stack}
IPv6 Single StackとはIPv6のみで動作させる仕組みである。IPv6 Single Stack環境構築にはIPv6 over IPv4トンネリングを利用する。
IPIPトンネリングとはIPv6パケット全体を一つのデータとして扱い、そのデータの先頭にIPv4ヘッダを付加しカプセル化することでIPv6ネットワークからIPv4ネットワークを
トンネリングし、IPv6通信を実現させる通信方法である。

\subsubsection{IPv4/IPv6 共存技術}
IPv4/IPv6共存環境にはDual Stack環境やNAT64/DNS64が存在する。
Dual Stackは現在一般的に用いられているIPv4/IPv6共存技術であり、単一の機器においてIPv4とIPv6を同時に動作させる事ができる。
IPv4対応危機と通信を行う際にはIPv4を使用し、IPv6対応機器と通信を行う際にはIPv6を使用する。一方NAT64/DNS64はIPv6ネットワークからIPv4ネットワーク
に接続する技術である。Dual StackがIPv4アドレスとIPv6アドレスの両方を必要とするのに対し、NAT64/DNS64はIPv6アドレスのみを与え利用される技術である。
そのため、Dual StackがIPv4アドレス枯渇問題対策として有効である。

Dual Stack環境は、RAを動作させているルータマシンでDHCPサーバのソフトウェアをインストールし起動させプライベートIPアドレス等必要な設定を行い、NATでIPv4ネットワーク
アクセス
する。NATとはプライベートIPv4アドレスとグローバルIPv4アドレスの間でアドレス変換をする技術である。
\subsubsection{シェルスクリプト}
シェルスクリプトとは、複数のUNIXコマンドを一つのファイルに改行区切りで記述したものである。
シェルスクリプトを作成し実行を行うと、ファイルに記載されているコマンドが上から順に実行される。

\subsubsection{ハードコーディング}
ソースコード上に直接書き込んでいけないものを直接ソースコード上に書き込んでしまうこと。
\newpage
\section{関連研究} %先に書く
\subsection{北口らによる調査}
北口らは、IPv6対応状況の研究としてOSの観点から調査を行なった。そこでは各種OSにおけるIPv6実装状況の検証がされ、さらにネットワーク運用管理に与える影響について
考察を行なっていた。表2は各OSにおけるSLAACで用いるインターフェースIDの生成手法、RAにおけるRDNSSオプションの実装、DHCPv6の実装及びIPv6 Single Stack環境での検証結果を
示している。ただしMicrosoft社のインターフェースIDの生成方法は確認できていない。

Android OSについては、バージョン4、5、6、7を対処に調査が行われており、全てのバージョンでDHCPv6に未対応であることと、バージョン4ではIPv6 Single Stack環境では動作しないことが示されている。
また運用管理の課題としては、セキュリティインシデント時のトレーサビリティ困難性が挙げられている。[2]
\subsection{Durdagiらによる調査}
DurdagiらはIPv6とIPv4のセキュリティと脅威の比較調査を行なった。しかし、技術使用の視点からの指摘はされているが、運用の視点や混在環境での脅威といった視点では議論は行われいない。[3]

\subsection{Rosilらによる調査}
RosliらはIPv6環境の脅威に焦点をあて、セキュリティ対策の提案をした。しかし、技術使用の視点からの指摘はされているが、運用の視点や混在環境での脅威といった視点では議論は行われていない。[4]


\subsection{加茂らよる調査}
加茂らはAndroidに焦点を当てIPv6 Single Stack環境でOSが稼働するか、マーケットが対応しているか、Androidアプリが動くか複数の視点で行われた。
しかし、Androidアプリの未対応である原因については今後の課題とされていた。[5]	

\newpage
\section{調査及び分析内容・手法}
アプリ内のIPアドレスのハードコーディングがあるのかを確認するために、APKをSmaliファイルに変換しSmaliファイル内にIPアドレスのハードコーディングがあるのかを確認する。
\subsection{APKからSmaliファイルの変換プログラム}
Smaliファイルの取得はapktoolを用いてapkファイルの変換を行う。
APKからSmaliファイルの変換を行うために以下のコマンドを行う。

コマンド内の\$APK\_FILEはファイルのpathを指定しています。また\/ssd1\/\$name\/\$OUTPUT\_FOLDERも同様です。

\begin{lstlisting}[caption=APKからSmaliファイルの変換のコマンド]
	apktool d $APK_FILE -o /ssd1/$name/$OUTPUT_FOLDER
\end{lstlisting}

今回は、多くのAPKファイルを調査する必要であるため、以上のコマンドを複数回実行する必要がある。
そのため、以上のコマンドをシェルスクリプトにまとめて実行することを推奨する。
複数のAPKをまとめてSmaliファイルの変換を行うためのシェルスクリプトを以下に記載する。

スクリプト内にある/media/\$(name)/*.apkは変換したいAPKのpathを指定しています。
また、\$(basename \$(APK\_FILE)\.apk)は変換するAPKのファイル名を取得しています。
\begin{lstlisting}[caption=複数のAPKからSmaliファイルの変換を行うシェルスクリプト]
#!/bin/sh
name = "フォルダ"
APK_FILES = $(find /media/$(name)/*.apk)
for APK_FILE in $APK_FILES
do
	echo $APK_FILE
	OUTPUT_FOLDER=$(basename $(APK_FILE).apk)
	apktool d $APK_FILE -o /ssd1/$name/$OUTPUT_FOLDER
done
\end{lstlisting}
シェルスクリプトを実行するためには以下のコマンドを実行した。

./Smali1.shは、Listing2で書き込んだスクリプトファイル名である。
\begin{lstlisting}[caption=シェルスクリプトの実行方法]
bash ./Smali1.sh 
\end{lstlisting}
以上でAPKをSmaliファイルに変換を行うことができる。

\subsection{Smaliファイル内からIPv4アドレスの探索プログラム}
Smaliファイル内からIPv4アドレスの探索を行うためのスクリプトをPythonで作成した。
11行目は正規表現の文字パターンを記述した。
18行目globモジュールを使うことで指定されたパターンに合うするファイルパス名を取得している。。
また、21行目のopen関数を使いglobで指定したファイルパスのファイルを開いている。
26行目から31行目で正規表現を用いてsmaliファイル内にIPv4アドレスの記載があるかを確認している。
38行目から39行目でIPv4アドレスの記載が見つかったファイルパスをresult\_pathに記述している。
\begin{lstlisting}[caption=Smaliファイル内からIPv4アドレスの探索を行うためのPythonスクリプト]
import glob
import os
import pprint #結果を見やすくするためのモジュール
import sys
import pathlib
import re   #正規表現のためのモジュール
import csv
import sys,os

def Find_IPv4(apk_file):
	args = '(([1-9]?[0-9]|1[0-9]{2}|2[0-4][0-9]|25[0-5])\.){3}([1-9]?[0-9]|1[0-9]{2}|2[0-4][0-9]|25[0-5])'
	count =0
	result_path = "/ssd1/IP_result/IP_File/"+apk_file+".csv"
	print("通過します")
	path = "/ssd1/"+apk_file+"/*/smali*/**/*.smali"
	result = pathlib.Path(result_path)
	result.touch()
	smail_files = glob.glob(path,recursive = True)
	pprint.pprint(args)
	for file_path in smail_files:
		with open(file_path,encoding="utf-8_sig") as f:
			lines = f.readlines()
			txtcount=0
			for contents in lines:
				txtcount +=1
				match2 = re.search(r'\s'+args+r'\s',contents)
				match3 = re.search(r'^'+args+r'$',contents)
				match4 = re.search(r'\"'+args+r'\"',contents)
				match5 = re.search(r"\s"+args+r"$",contents)
				match6 = re.search(r"^"+args+r"\s",contents)
				match7 = re.search(r"'"+args+r"'",contents)
				contents = f.readline()
				if match2 or match3 or match4 or match5 or match6 or match7:
					pprint.pprint(file_path)
					pan = [file_path,str(txtcount)]
					count += 1
					with open(result_path,"a", newline='') as n:
						writer = csv.writer(n, lineterminator='\n')
						writer.writerow(pan)
					n.close
				else:
					continue
		f.close
	print("Smali内のIPv4探索を終了します")
	return(result_path)
		
\end{lstlisting}

\subsection{グローバルIPアドレスとプライベートIPアドレスの割合調査プログラム}
グローバルIPアドレスとプライベートIPアドレスの割合調査するPythonスクリプトを作成した。

319行目でプライベートIPアドレスの正規表現パターンを記述して、マッチしなかったものをグローバルIPアドレスと判定している。

\begin{lstlisting}
import csv
import pprint
import re
def Distinction(txt,IP_Adress):
	ip_list =[]
	ip_list2= []
	#プライベートIPアドレス
	file_list=[]
	#グローバルIPアドレス
	file_list2=[]
	PrivateIPv4 =0
	GlobalIPv4 = 0
	args = '(^127\.)|(^169\.254\.)|(^10\.)|(^172\.1[6-9]\.)|(^172\.2[0-9]\.)|(^172\.3[0-1]\.)|(^192\.168\.)'
	with open(IP_Adress,encoding="utf-8") as n:
		ip_adress = csv.reader(n)
		for ip in ip_adress:
			ip_list.append([ip[0]])
	n.close
	with open(txt, encoding="utf-8") as f:
		reader = csv.reader(f)
		for row in reader:
			file_name = row[0].split('/smali')[0]
			a =row[2].strip('"')
			ip_name = a.strip()
			ip_list2.append([file_name,ip_name])
	f.close
	for ip in ip_list:
		for ans in ip_list2:
			#同じIp_Adressならば通過できる
			if ip[0] == ans[1]:
				#print(file_list)
				print(ans[0])
				match = re.search(args,ans[1])
				if match:
					if ans[0] not in file_list:
						print("プライベートIPアドレスあったよ")
						PrivateIPv4+=1
						file_list.append(ans[0])
				else:
					if ans[0] not in file_list2:
						print("グローバルIPアドレス")
						GlobalIPv4+=1
						file_list2.append(ans[0])
	return
	
\end{lstlisting}

\subsection{Androidアプリ内にIPv4アドレスのハードコーディング調査}
Androidアプリ内にIPv4アドレスのハードコーディング調査のため、
複数のAPKを対象にSmali化を行いSmaliファイル内にIPアv4ドレスの記載があるのか調査した。
\subsection{グローバルIPアドレスとプライベートIPアドレスの割合調査}
IPアドレスにはグローバルIPアドレスとプライベートIPアドレスの2つがある。
グローバルIPアドレスはインターネットを利用するときに割り振られるIPアドレス
でセキュリティ上の問題でプロバイダ側が一定時間などで変更を行なっている場合がある。
一方、プライベートIPアドレスはインターネットのようなグローバルなものではなく、社内LANなど小規模なネットワークで使用されるIPアドレスです。
特に、グローバルIPアドレスのハードコーディングがある場合攻撃者によって中間者攻撃やDDoS攻撃など悪用されるリスクが高まる
と考えられる。また、プライベートIPアドレスのハードコーディングがある考えられる理由として、
誤って記述してしまった場合やテストコード用に記述したものなどが挙げられるが、
プライベートIPアドレスの固定で指定するのは通常動作としては考えにくいためグローバルIPアドレスとプライベートIPアドレスの割合を調査した。

\newpage
\section{結果}
\subsection{Androidアプリ内にIPアドレスのハードコーディングがされているのか調査結果}
調査を行なったアプリ数109251個の中45305個のアプリにIPv4アドレスの記載があり、全体の41.4\%にIPアドレスの記載がある事が分かった。
以下の表に、出現数の多いIPアドレス別の結果をまとめた。
\begin{table}[htb]
  \begin{center}
    \caption{最も多いIPアドレス1〜20}
    \begin{tabular}{|r|r|r|} \hline
      IPアドレス & アプリ数 & アドレス区分け\\ \hline \hline
10.0.0.172 & 35151 &プライベートアドレス\\ \hline
127.0.0.1 & 29165& ループバックアドレス\\ \hline
10.0.0.200 & 20794 & プライベートアドレス\\ \hline
0.0.0.0 & 14177 & ループバックアドレス\\ \hline
117.97.87.6 & 4503 & グローバルアドレス\\ \hline
127.0.0.255 & 4202 & ループバックアドレス\\ \hline
2.5.29.15 & 3718 & グローバルアドレス\\ \hline
2.5.4.3 & 3197 & グローバルアドレス\\ \hline
2.5.4.11 & 3190& グローバルアドレス \\ \hline
2.5.4.6 & 3188 & グローバルアドレス\\ \hline
2.5.4.10 & 3187 & グローバルアドレス\\ \hline
2.5.4.8 & 3186 & グローバルアドレス\\ \hline
2.5.4.7 & 3183 & グローバルアドレス\\ \hline
2.5.29.37 & 3065 & グローバルアドレス\\ \hline
10.0.2.2 & 2878 & プライベートアドレス\\ \hline
2.5.29.19 & 2876& グローバルアドレス \\ \hline
1.4.1.19 & 2762 & グローバルアドレス\\ \hline
2.0.61.0 & 2758 & グローバルアドレス\\ \hline
2.5.4.4 & 2360 & グローバルアドレス\\ \hline
2.5.4.42 & 2354 & グローバルアドレス\\ \hline

    \end{tabular}
    \label{tab:price}
  \end{center}
\end{table}

\newpage

\subsection{グローバルIPアドレスとプライベートIPアドレスの記載割合調査結果}

調査したアプリ数とグローバルIPアドレスの記載があるアプリ数、プライベートIPアドレスの記載があるアプリ数との記載割合とIPアドレスの記載があるアプリとグローバルIPアドレスがあるアプリ数、
プライベートIPアドレスの記載があるアプリ数との記載割合を次のような表2〜3でまとめた。

\begin{table}[htb]
\begin{center}
	\caption{IPアドレスの記載率}
	\begin{tabular}{|r|r|r|r|r|}

	\hline
	調査したアプリ数 & グローバルIPアドレス & 記載率 & プライベートIPアドレス&記載率 \\
	&記載があるアプリ数 & &記載があるアプリ数 &\\
	\hline
		109251 & 29074 & 26.6\% & 35151&29.7\% \\
	\hline
	\end{tabular}
	\end{center}
\end{table}

\begin{table}[htb]
	\begin{center}
		\caption{グローバルアドレスとプライベートアドレスの割合}
		\begin{tabular}{|r|r|r|r|r|}
	
		\hline
		IPアドレスの記載が& グローバルIPアドレス & 割合率 & プライベートIPアドレス&割合率 \\
		があるアプリ数&記載があるアプリ数 & &記載があるアプリ数 &\\
		\hline
			45305 & 29074 & 64.1\% & 35151&77.5\% \\
		\hline
		\end{tabular}
		\end{center}
	\end{table}


\newpage
\section{考察}
\subsection{Androidアプリ内にIPアドレスのハードコーディング調査の分析}
アプリの約4割にIPアドレスの記載があるアプリがあり、グローバルIPアドレスの記載があるアプリには
プライベートIPアドレスの記載がされている可能性があると考察できる。
また、IPアドレスの記載があるアプリの半数に”127.0.0.1”や"127.0.0.255"、"0.0.0.0"などのループバックアドレスが使われていることが分かる。
これは、アプリ上で何かしらの理由でローカルマシンにアクセスするために記載されているのではないかと考察ができる。
また、表1を見ると"2.5.*.*"が記載されている。これは、AWSなどのサーバー管理会社にアクセスしているものなのではないかと考察ができる。


\subsection{IPアドレスのハードコーディングにおける脅威}
アプリ内にIPアドレスのハードコーディングがされている場合、どのような脅威が存在するか考察する。
まず、悪意ある者からのアプリ内に記載されているIPアドレスに向けてDDoS攻撃が挙げられる。アプリ内にIPアドレスのハードコーディングがされている場合、
悪意ある者がアプリ内に記載されているIPアドレスに向けて大量のパケット送り、攻撃対象の機能を停止させ、サービス提供をやめさせるなどの可能性がある。

また、悪意ある者からの中間者攻撃が挙げられる。
悪意ある者が攻撃対象者と同じネットワークに入り込み、通信の中間に入ることによって
攻撃対象者にバレずに情報を抜き取られてしまう可能性がある。
例として、悪意ある者が脆弱性のあるアプリを利用し、通信先を変えたり情報を取得されたりするものがある。
\newpage
\section{まとめ}
本研究はAndroidアプリ内にIPv4アドレスのハードコーディングが行われていた際の脅威を検討し、
Smaliファイル内にIPv4アドレスのハードコーディング調査の必要性を示した。
次に、本研究の分析内容と手法を示したうえで、
Smaliファイル内にIPv4アドレスのハードコーディング調査を行い結果を示した。
今回、Androidアプリ109251個を調査したうち45305個にIPv4アドレスのハードコーディングが行われており全体の41.4\%であった。
このことから、Androidアプリ内にIPv4アドレスのハードコーディングが行われていることが分かった。
また、最も出現数の多いIPv4アドレスに"2.5.*.*"のグローバルIPアドレスが多く存在した。
このことから、サーバー管理会社などにアクセスしているものなのではないかと考察した。
調査結果をもとに、Androidアプリ内にIPv4アドレスのハードコーディングが行われるリスクを考察し、そこから起こる脅威や分析を考察した。
\newpage
\begin{thebibliography}{99}

\bibitem{one}Support\-Apple Developer, "Supporting Ipv6\-only Netwoorks", 2016, https:\/\/developer.apple.com\/suppor\/ipv6

\bibitem{two}JariArikko, Arikeranen, “Experiences from an IPv6\-only Network“, 2012,https:\/\/tools.itef.org\/html\/rfc6586

\bibitem{three}北口 善明, 近堂 徹, 鈴田 伊知郎, 小林 貴之, 前野 譲二, “クライアント OS の IPv6実装検証から見たネットワーク運用における課題の考察”, デジタルプラクティス, 2018

\bibitem{four}加茂恵梨香 Android環境のIPv6対応の調査と分析
\end{thebibliography}

\end{document}