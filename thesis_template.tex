\documentclass[a4j]{jarticle}
\date{}
\usepackage[dvipdfmx]{graphicx}

\begin{document}


%表題
\makeatletter %ここから\makeatotherまで触らなくていい
	\def\@thesis{平成28年度 東邦大学理学部情報科学科 卒業研究}
	\def\id#1{\def\@id{#1}}
	\def\department#1{\def\@department{#1}}
	
	\def\@maketitle{
		\begin{center}
			\vspace{10mm}
			{\large \@thesis \par}	%修士論文と記載される部分
			\vspace{50mm}
			{\huge\bf \@title \par}	% 論文のタイトル部分
			\vspace{15mm}
			{\Large 学籍番号 \@id \par}	% 学籍番号部分
			\vspace{5mm}
			{\Large \@author \par}	% 氏名
			\vspace{50mm}
		\end{center}
		\begin{flushright}
			{\large 金岡研究室}
		\end{flushright}
	}
\makeatother

\title{タイトルです} %中括弧の中に卒論のタイトル
\id{5517044} %学籍番号
\author{小林 裕} %名前
\maketitle{\title} %表紙の出力
\thispagestyle{empty} %このページ(表紙)のページ番号を消去
\newpage %強制改ページ

\tableofcontents %目次の出力





\newpage
\section{はじめに} 




\newpage
\section{前提知識}
\subsection{Android}

ああああああああ
かかかかかかかか

あああああああああああああああああああああああああああああああああああああああああああああああああああああああああああああああああああああああああああああああああああああああああ
\subsubsection{アプリケーション配布マーケット}
\subsubsection{アプリ開発}
\subsubsection{ライブラリ}
\subsubsection{APK}
\subsubsection{APKストア}
\subsubsection{Smaliファイル}
\subsubsection{APKからSmaliファイルの変換}
\begin{itemize}
\item まずは
\item この章のなかで書くことを
\item 箇条書きで書き出してみる
\item ことから始めましょう
\end{itemize}

\subsection{IP}
あああああ
あああああ
\subsubsection{IPv4アドレス枯渇問題}
\subsubsection{IPv6 Single Stack}
\subsubsection{IPv4/IPv6 共存環境}
\begin{itemize}
\item まずは
\item この章のなかで書くことを
\item 箇条書きで書き出してみる
\item ことから始めましょう
\end{itemize}








\newpage
\section{関連研究} %先に書く
\subsection{北口らによる調査}

\subsection{Durdagiらによる調査}

\subsection{Rosilらによる調査}

\subsection{加茂らよる調査}


\newpage
\section{調査及び分析内容・手法}
\subsection{調査環境のデザイン}
\subsubsection{APKからSmaliファイルに変換プログラム}
あああああ
あああああ
\begin{itemize}
\item まずは
\item この章のなかで書くことを
\item 箇条書きで書き出してみる
\item ことから始めましょう
\end{itemize}
\subsubsection{Smaliファイル内からIPv4アドレスの探索プログラム}
\subsection{アプリ調査}

\newpage
\section{調査結果}
\subsection{アプリ調査結果}
あああああ
あああああ

\begin{itemize}
\item まずは
\item この章のなかで書くことを
\item 箇条書きで書き出してみる
\item ことから始めましょう
\end{itemize}

\newpage
\section{考察}
\subsection{Androidアプリ内にIPアドレスのハードコーディング有無}
\subsection{IPアドレスのハードコーディングにおける脅威}
\begin{itemize}
\item まずは
\item この章のなかで書くことを
\item 箇条書きで書き出してみる
\item ことから始めましょう
\end{itemize}

\newpage
\section{まとめ}
あああああ

あああああ

\begin{itemize}
\item まずは
\item この章のなかで書くことを
\item 箇条書きで書き出してみる
\item ことから始めましょう
\end{itemize}



\newpage
\begin{thebibliography}{99}
\bibitem{bib01}
だれだれ, "文献1", 年度
\bibitem{bib02}
だれだれ, "文献2", 年度
\bibitem{bib03}
だれだれ, "文献3", 年度
\bibitem{bib04}
だれだれ, "文献4", 年度
\bibitem{bib05}
だれだれ, "文献5", 年度
\bibitem{bib06}
だれだれ, "文献6", 年度
\bibitem{bib07}
だれだれ, "文献7", 年度
\bibitem{bib08}
だれだれ, "文献8", 年度
\bibitem{bib09}
だれだれ, "文献9", 年度
\bibitem{bib010}
だれだれ, "文献10", 年度

\end{thebibliography}

\end{document}